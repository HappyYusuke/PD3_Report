\section{実験目的}
本プロジェクトでは、雑多な環境下においてロボットの最大直進速度で追従できる人追従システムの
開発を目的としている。これに伴った実験の目的は、雑多な環境下での人追従の制度とロボットの
最大直進速度での人追従性能の2つの検証をすることである。以上のことから、追従実験と最大
追従速度実験により、開発した人追従システムの性能を検証する。

\section{実験方法}
実験では、雑多な環境を作成し追従実験と最大追従速度実験をする。
実験中は、人は追従対象の1人のみとする。\\ \indent
要求仕様(2)を検証するため、追従実験では直線経路、曲線経路、直角経路をそれぞれ10回実験し、
成功率を算出する。追従の成功率が各経路において90\%以上であった場合に要求仕様(2)を満たしたものとする。
また、要求仕様(3)を検証するため、10[m]以上の直線経路にて最大追従速度実験をする。
0.1[m/s]から、0.1ずつ速度を上昇させ、追従できなくなる速度の直前を最大追従速度とする。
Turtlebot3 Big Wheelの最大直進速度が0.5[m/s]であるため、開発した人追従システムの
最大追従速度が0.5[m/s]であった場合に要求仕様(3)は満たされる。
以上の実験は、2D-LiDARのデータを用いた実験であるため、要求仕様(2)、(3)が満たされたら、
要求仕様(1)も満たされたものとする。

\subsection{実験機材}
本プロジェクトでは、2023年に開催されたRoboCup2023で開発したHappy Eduを使用する。
Happy Eduの全体像をFig. \ref{Happy Edu}に示す。Happy Eduのロボット台車には、
ROBOTIS社の二輪差動駆動台車であるTurtleBot3 Big Wheelを用いてる。2D-LiDARには、
北洋電機株式会社のUTM30-LXを用いている。ロボットに搭載するノートPCは、ASUSの
ROG Strix G16を用いている。本プロジェクトでは、ロボット台車、2D-LiDAR、ロボット用
ノートPCの3つを使用するため、Table \ref{Happy Edu specifications}にそれぞれの仕様
を示す。

\begin{figure*}[h]
  \begin{center}
  \includegraphics[width=60mm,clip]{figure/Happy_Edu.jpg}
  \caption{Happy Edu}
  \label{Happy Edu}
  \end{center}
\end{figure*}

\begin{table}[h]
  \begin{center}
    \caption{{Happy Edu specifications}\label{Happy Edu specifications}}
    \scalebox{0.8}[0.9]{
      \begin{tabular}{c|c|c} \hline
        Type of equipment & Product name & Specifications \\ \hline
        Robot base & TurtleBot3 Big Wheel & 
        \begin{tabular}{c}
        Maximum straight speed: 0.6[m/s]\\
        Maximum rotation speed: 3.14[rad/s]\\
        Maximum payload: 30[kg]\\
        Size(length x width x height): 281mm x 306mm x 170.30 [mm]
        \end{tabular} \\ \hline
        Robot laptop & ASUS ROG Strix G16 &  
        \begin{tabular}{c}
        CPU: Corei7 13650HX\\
        GPU: GeForce RTX4070 8GB\\
        Weight: 2.5 [kg]\\
        Size: 26.4 x 35.4 x 2.26 [cm]\\
        \end{tabular} \\ \hline
        2D-LiDAR & UTM-30LX & 
        \begin{tabular}{c}
        Maximum detection distance: 60 [m]\\
        Scan range: 270 [deg]\\
        Angle resolution energy: 0.25 [deg]\\
        Scan time: 25 [ms]
        Weight: 210 [g]
        \end{tabular} \\ \hline
      \end{tabular}
    }
  \end{center}
\end{table}

\subsection{追従実験}
要求仕様(2)を検証するため、
Fig. \ref{Image of tracking experiment environment (FMT Laboratory Room 206)}と
Fig. \ref{Image of tracking experiment environment (FMT Laboratory Room 326)}
のような雑多な環境を用意し、追従の成否を実験する。
雑多な環境の用意では、人の脚部と類似している椅子やポールなどの円柱状の物体を多く
設置している。\\ \indent
直線経路、曲線経路、直角経路のイメージをそれぞれ
Fig. \ref{Straight road}、Fig. \ref{Curved road}、Fig. \ref{Right angle road}に示す。
直線経路、曲線経路、直角経路においてそれぞれ10回ずつ試行し、各径路での追従成功率を算出する。
追従が成功した回数を$x_{success}$とした時の追従成功率$success \; rate$は以下のようになる。

\begin{equation}
\label{Success rate}
  success \; rate = x_{success} / 10
\end{equation}

(\ref{Success rate})式を用いて各径路での成功率を算出し、それぞれの成功率が90[\%]以上
であれば要求仕様(2)を満たしたこととする。

\begin{figure*}[h]
  \begin{center}
  \includegraphics[width=90mm,clip]{figure/experimental_env1.JPG}
  \caption{Image of tracking experiment environment (FMT Laboratory Room 206)}
  \label{Image of tracking experiment environment (FMT Laboratory Room 206)}
  \end{center}
\end{figure*}

\begin{figure*}[h]
  \begin{center}
  \includegraphics[width=90mm,clip]{figure/experimental_env2.png}
  \caption{Image of tracking experiment environment (FMT Laboratory Room 326)}
  \label{Image of tracking experiment environment (FMT Laboratory Room 326)}
  \end{center}
\end{figure*}

\begin{figure*}[h]
  \begin{center}
  \includegraphics[width=100mm,clip]{figure/Straight.PNG}
  \caption{Straight road}
  \label{Straight road}
  \end{center}
\end{figure*}

\begin{figure*}[h]
  \begin{center}
  \includegraphics[width=100mm,clip]{figure/Curve.PNG}
  \caption{Curved road}
  \label{Curved road}
  \end{center}
\end{figure*}

\begin{figure*}[h]
  \begin{center}
  \includegraphics[width=100mm,clip]{figure/Right-angle.PNG}
  \caption{Right angle road}
  \label{Right angle road}
  \end{center}
\end{figure*}

\clearpage

\subsection{最大追従速度実験}
要求仕様(3)を検証するため、
Fig. \ref{Maximum tracking speed experiment environment image (FMT Laboratory Room 326)}
のような10[m]の直線経路を用意し最大追従速度を計測する。実験する速度は0.1[m/s]から開始し、
0.1ずつ速度を上昇させ、Happy Eduが追従できなくなった場合の1つ前の速度を最大追従速度とする。
追従の成否は、Happy Eduが追従対象者を10[m]以上追従したことを追従成功とする。
また、Happy Eduの追従速度は追従対象者の歩行速度に依存しており、
追従対象者の歩行速度を指定した速度にする必要がある。
そのため、Fig. \ref{Image of maximum tracking speed experiment}のような構成で実験
する。0.1[m/s]の場合であれば、Fig. \ref{Image of maximum tracking speed experiment}
の「Speed keeper」を0.1[m/s]で直進させ、「Speed keeper」に追従対象者が追従する。さらに、
追従対象者にHappy Eduが追従することで、指定した速度での実験をする。これを、0.1[m/s]から
開始し、Happy Eduが追従できなくなるまで試行を繰り返す。

\begin{figure*}[h]
  \begin{center}
  \includegraphics[width=80mm,clip]{figure/experimental_env3.JPG}
  \caption{Maximum tracking speed experiment environment image (FMT Laboratory Room 326)}
  \label{Maximum tracking speed experiment environment image (FMT Laboratory Room 326)}
  \end{center}
\end{figure*}

\begin{figure*}[h]
  \begin{center}
  \includegraphics[width=90mm,clip]{figure/max_experiment_image.JPG}
  \caption{Image of maximum tracking speed experiment}
  \label{Image of maximum tracking speed experiment}
  \end{center}
\end{figure*}

\clearpage

\section{実験結果}
\subsection{追従実験}
\begin{figure*}[h]
  \begin{center}
  \includegraphics[width=100mm,clip]{figure/Tracking-experiment-Real-view.jpg}
  \caption{Tracking experiment (Real view)}
  \label{Tracking experiment (Real view)}
  \end{center}
\end{figure*}

\begin{figure*}[h]
  \begin{center}
  \includegraphics[width=170mm,clip]{figure/Tracking-experiment-Internal-view.png}
  \caption{Tracking experiment (Internal view)}
  \label{Tracking experiment (Internal view)}
  \end{center}
\end{figure*}

\begin{table}[h]
    \begin{center}
      \caption{{Success rate of traking in each road}\label{Success rate of traking in each road}}
      \scalebox{1.2}[1.0]{
        \begin{tabular}{c|r} \hline
          Road & Success rate [\%] \\ \hline
          Straight road & 100 \\
          Curved road & 100 \\
          Right angle road & 100 \\ \hline
        \end{tabular}
      }
    \end{center}
\end{table}

直線経路、曲線経路、直角経路、最大追従速度の実験結果をFig. \ref{Result}に示す。
追従実験では、直線経路と直角経路がそれぞれ10回中10回成功した。曲線経路では、10回中9回成功した。

\subsection{最大追従速度実験}

\begin{table}[h]
  \begin{center}
    \caption{{Maximum tracking speed experimental result}\label{Maximum tracking speed experimental result}}
    \scalebox{1.0}[0.9]{
      \begin{tabular}{c|r} \hline
        Tracking speed [m/s] & Travel distance tracked [m] \\ \hline
        0.1 & 11.73 \\
        0.2 & 11.31 \\
        0.3 & 11.58 \\
        0.4 & 11.53 \\
        0.5 & 11.48 \\
        0.6 & 6.231 \\ \hline
      \end{tabular}
    }
  \end{center}
\end{table}

\begin{figure*}[h]
  \begin{center}
  \includegraphics[width=150mm,clip]{figure/Maximum-tracking-speed-experimental-result.png}
  \caption{Tracking speed graph}
  \label{Tracking speed graph}
  \end{center}
\end{figure*}

最大追従実験では、0.5[m/s]が最大追従速度となった。0.6[m/s]で3回実験したが、追従は確認されなかった。

\section{考察}
実験結果から直線経路と直角経路では、雑多な環境において一度も止まらず安定した挙動で要求仕様を
満たすことができた。しかし、曲線経路では一度だけ停止した。曲線経路で停止した原因は、ロボット用ノートPCのバッテリー低下だと考えられる。ロボットが停止しなかった場合のROS2 bagでは、システム全体の通信が平均10[Hz]以上で安定しているのに対し、曲線経路で停止したときは、平均4.9[Hz]であった。実験が成功しているROS2 bagでは、通信速度の低下は確認されず、ノートPCのバッテリーが10\%以下であったのは、曲線経路で停止した場合のみであった。