%ここからソースコードの表示に関する設定
\lstset{
  basicstyle={\ttfamily},
  identifierstyle={\small},
  commentstyle={\smallitshape},
  keywordstyle={\small\bfseries},
  ndkeywordstyle={\small},
  stringstyle={\small\ttfamily},
  frame={tb},
  breaklines=true,
  columns=[l]{fullflexible},
  numbers=left,
  xrightmargin=0zw,
  xleftmargin=3zw,
  numberstyle={\scriptsize},
  stepnumber=1,
  numbersep=1zw,
  lineskip=-0.5ex
}
%ここまでソースコードの表示に関する設定
パラメータの設定ファイルを、ソースコード\ref{param}に示す。
\begin{lstlisting}[caption=follow\_me\_params.yaml, label=param]
  /follow_me/laser_to_img:
    ros__parameters:
      # 縮小サイズを取得. 1[px] = 0.01[m]
      discrete_size: 0.01
      # Max LiDAR Range
      max_lidar_range: 3.5
      # 画像を表示するフラッグ
      img_show_flg: False
  
  /follow_me/person_detector:
    ros__parameters:
      # 追従対象者との距離
      target_dist: 0.5
      # 追従対象を見失ったときに追従を再開する時の距離の誤差
      target_diff: 0.3
      # 追従ポイント(制御を止める領域)の半径
      target_radius: 0.1
      # 人を検出する範囲(円)の半径
      target_range: 0.4
      # 起動時に追従対象者を検出するまでの待機時間
      init_time: 3.0
      # 起動時に追従対象を検出するまでのflg
      none_person_flg: True
    
  /follow_me/base_controller:
    ros__parameters:
      # ロボットからみてtolerance[°]以内だったら積分制御しない視野角
      tolerance: 1.0
      # 積分制御をし始める視野角[°]
      i_range: 3.0
      # 並進のPゲイン==========================
      lkp: 0.3
      # 旋回のPIDゲイン========================
      # Pゲイン
      akp: 0.005
      # Iゲイン
      aki: 0.0
      # Dゲイン
      akd: 0.0009
\end{lstlisting}

2D-LiDARの距離データから俯瞰画像を生成するソースコードを、ソースコード\ref{image}に示す。
\begin{lstlisting}[caption=laser\_to\_image.py, label=image]
    import numpy as np
    import os
    import sys
    import cv2
    import math
    import rclpy
    from rclpy.node import Node
    from rclpy.parameter import Parameter
    from sensor_msgs.msg import LaserScan, Image
    from rclpy.qos import qos_profile_sensor_data
    from rcl_interfaces.msg import SetParametersResult
    from cv_bridge import CvBridge, CvBridgeError
    # Custom
    from .modules.gradient import gradation_3d_img as gradation
    
    
    class LaserToImg(Node):
        def __init__(self):
            super().__init__('laser_to_img')
            # Publisher
            self.pub = self.create_publisher(Image, '/follow_me/laser_img', 10)
            # Subscriber
            self.create_subscription(LaserScan, '/scan', self.cloud_to_img_callback, qos_profile_sensor_data)
            # OpenCV
            self.bridge = CvBridge()
            # Parameters
            self.declare_parameters(
                    namespace='',
                    parameters=[
                        ('discrete_size', Parameter.Type.DOUBLE),
                        ('max_lidar_range', Parameter.Type.DOUBLE),
                        ('img_show_flg', Parameter.Type.BOOL)])
            self.add_on_set_parameters_callback(self.param_callback)
            # Get parameters
            self.param_dict = {}
            self.param_dict['discrete_size'] = self.get_parameter('discrete_size').value
            self.param_dict['max_lidar_range'] = self.get_parameter('max_lidar_range').value
            self.param_dict['img_show_flg'] = self.get_parameter('img_show_flg').value
            # Values
            self.color_list = gradation([0,0,255], [255,0,0], [1, 100], [True,True,True])[0]
            # Output
            self.output_screen()
    
        def output_screen(self):
            for key, value in self.param_dict.items():
                self.get_logger().info(f"{key}: {value}")
    
        def param_callback(self, params):
            for param in params:
                self.param_dict[param.name] = param.value
                self.get_logger().info(f"Set param: {param.name} >>> {param.value}")
            return SetParametersResult(successful=True)
    
        def cloud_to_img_callback(self, scan):
            
            # discrete_factor
            discrete_factor = 1/self.param_dict['discrete_size']
            # max_lidar_rangeとdiscrete_factorを使って画像サイズを設定する
            img_size = int(self.param_dict['max_lidar_range']*2*discrete_factor)
    
            # LiDARデータ
            maxAngle = scan.angle_max
            minAngle = scan.angle_min
            angleInc = scan.angle_increment
            maxLength = scan.range_max
            ranges = scan.ranges
            intensities = scan.intensities
            #intensities = scan.intensities
            
            # 距離データの個数を格納
            num_pts = len(ranges)
            # 721行2列の空行列を作成
            xy_scan = np.zeros((num_pts, 2))
            # 3チャンネルの白色ブランク画像を作成
            blank_img = np.zeros((img_size, img_size, 3), dtype=np.uint8) + 255
            # rangesの距離・角度からすべての点をXYに変換する処理
            for i in range(num_pts):
                # 範囲内かを判定
                if (ranges[i] > self.param_dict['max_lidar_range']) or (math.isnan(ranges[i])):
                    pass
                else:
                    # 角度とXY座標の算出処理
                    angle = minAngle + float(i)*angleInc
                    xy_scan[i][1] = float(ranges[i]*math.cos(angle))  # y座標
                    xy_scan[i][0] = float(ranges[i]*math.sin(angle))  # x座標
    
            # ブランク画像にプロットする処理
            for i in range(num_pts):
                pt_x = xy_scan[i, 0]
                pt_y = xy_scan[i, 1]
                if (pt_x < self.param_dict['max_lidar_range']) or (pt_x > -1*(self.param_dict['max_lidar_range']-self.param_dict['discrete_size'])) or (pt_y < self.param_dict['max_lidar_range']) or (pt_y > -1 * (self.param_dict['max_lidar_range']-self.param_dict['discrete_size'])):
                    pix_x = int(math.floor((pt_x + self.param_dict['max_lidar_range']) * discrete_factor))
                    pix_y = int(math.floor((self.param_dict['max_lidar_range'] - pt_y) * discrete_factor))
                    if (pix_x > img_size) or (pix_y > img_size):
                        print("Error")
                    else:
                        blank_img[pix_y, pix_x] = [0, 0, 0]
    
            # CV2画像からROSメッセージに変換してトピックとして配布する
            img = self.bridge.cv2_to_imgmsg(blank_img, encoding="bgr8")
            self.pub.publish(img)
    
            # 画像の表示処理. imgshow_flgがTrueの場合のみ表示する
            if self.param_dict['img_show_flg']:
                cv2.imshow('laser_img', blank_img)
                cv2.waitKey(3)
                #更新のため一旦消す
                blank_img = np.zeros((img_size, img_size, 3))
            else:
                pass
    
    def main():
        rclpy.init()
        node = LaserToImg()
        try:
            rclpy.spin(node)
        except KeyboardInterrupt:
            pass
        node.destroy_node()
        rclpy.shutdown()
\end{lstlisting}

追従対象者を特定するソースコードを、ソースコード\ref{person}に示す。
\begin{lstlisting}[caption=person\_detector.py, label=person]
    import math
    import time
    import cv2
    import rclpy
    from rclpy.node import Node
    from rclpy.parameter import Parameter
    from rcl_interfaces.msg import SetParametersResult, ParameterEvent
    from rcl_interfaces.srv import GetParameters
    from sensor_msgs.msg import Image
    from geometry_msgs.msg import Point
    from cv_bridge import CvBridge, CvBridgeError
    from yolov8_msgs.msg import DetectionArray
    
    
    class PersonDetector(Node):
        def __init__(self):
            super().__init__('person_detector')
            # OpenCV Bridge
            self.bridge = CvBridge()
            # Publisher
            self.point_pub = self.create_publisher(Point, '/follow_me/target_point', 10)
            self.img_pub = self.create_publisher(Image, '/follow_me/image', 10)
            # Subscriber
            self.create_subscription(DetectionArray, '/yolo/detections', self.yolo_callback, 10)
            self.create_subscription(Image, '/yolo/dbg_image', self.img_show, 10)
            self.create_subscription(ParameterEvent, '/parameter_events', self.param_event_callback, 10)
            # Service
            self.srv_client = self.create_client(GetParameters, '/follow_me/laser_to_img/get_parameters')
            while not self.srv_client.wait_for_service(timeout_sec=0.5):
                self.get_logger().info('/follow_me/laser_to_img server is not available ...')
            # Parameters
            self.declare_parameters(
                    namespace='',
                    parameters=[
                        ('target_dist', Parameter.Type.DOUBLE),
                        ('init_time', Parameter.Type.DOUBLE),
                        ('none_person_flg', Parameter.Type.BOOL),
                        ('target_diff', Parameter.Type.DOUBLE),
                        ('target_radius', Parameter.Type.DOUBLE),
                        ('target_range', Parameter.Type.DOUBLE)])
            self.add_on_set_parameters_callback(self.param_callback)
            # Get parameters
            self.param_dict ={}
            self.param_dict['target_dist'] = self.get_parameter('target_dist').value
            self.param_dict['init_time'] = self.get_parameter('init_time').value
            self.param_dict['none_person_flg'] = self.get_parameter('none_person_flg').value
            self.param_dict['target_diff'] = self.get_parameter('target_diff').value
            self.param_dict['target_radius'] = self.get_parameter('target_radius').value
            self.param_dict['target_range'] = self.get_parameter('target_range').value
            self.param_dict['discrete_size'] = self.get_param()  # laser_to_imgからもってくる
            # Value
            self.person_list = []
            self.target_data = []  # 追従対象のデータを保存するリスト
            self.before_data = [0.0, 0.0, 0.0]
            self.center_x = 0.0
            self.center_y = 0.0
            self.target_point = Point()
            self.target_px = []
            self.laser_img = 0.0
            self.height = 0.0
            self.width = 0.0
            # Output
            self.output_screen()
    
        def output_screen(self):
            for key, value in self.param_dict.items():
                self.get_logger().info(f"{key}: {value}")
    
        def param_event_callback(self, receive_msg):
            for data in receive_msg.changed_parameters:
                if data.name == 'discrete_size':
                    self.param_dict['discrete_size'] = data.value.double_value
                    self.get_logger().info(f"Param event: {data.name} >>> {self.param_dict['discrete_size']}")
    
        def param_callback(self, params):
            for param in params:
                self.param_dict[param.name] = param.value
                self.get_logger().info(f"Set param: {param.name} >>> {param.value}")
            return SetParametersResult(successful=True)
        
        def get_param(self):
            req = GetParameters.Request()
            req.names = ['discrete_size']
            future = self.srv_client.call_async(req)
            while rclpy.ok():
                rclpy.spin_once(self, timeout_sec=0.1)
                if future.done():
                    break
            return future.result().values[0].double_value
    
        def yolo_callback(self, receive_msg):
            if not receive_msg.detections:
                self.center_x = self.center_y = None
            else:
                self.person_list.clear()
                for person in receive_msg.detections:
                    px = Point()
                    px.x = person.bbox.center.position.x
                    px.y = person.bbox.center.position.y
                    self.person_list.append(px)
    
        def plot_robot_point(self):
            # 画像の中心を算出
            robot_x = round(self.width / 2)
            robot_y = round(self.height / 2)
            # 描画処理
            cv2.circle(img = self.laser_img,
                       center = (round(robot_x), round(robot_y)),
                       radius = 5,
                       color = (0, 255, 0),
                       thickness = -1)
    
        def plot_person_point(self):
            cv2.circle(img = self.laser_img,
                       center = (round(self.center_x), round(self.center_y)),
                       radius = 8,
                       color = (0, 0, 255),
                       thickness = -1)
    
        def plot_target_point(self):
            cv2.circle(img = self.laser_img,
                       center = (int(self.target_px[0]), int(self.target_px[1])),
                       radius = int(self.param_dict['target_radius']/self.param_dict['discrete_size']),
                       color = (255, 0, 0),
                       thickness = 2)
    
        def plot_target_range(self):
            cv2.circle(img = self.laser_img,
                       center = (int(self.before_data[2][0]), int(self.before_data[2][1])),
                       radius = int(self.param_dict['target_range']/self.param_dict['discrete_size']),
                       color = (196, 0, 255),
                       thickness = 2)
    
        def diff_distance(self, data):
            return abs(data - self.before_data[0])
    
        def euclidean_distance(self, data, before_data):
            return math.sqrt((data.x-before_data.x)**2 + (data.y-before_data.y)**2)
    
        def select_target(self, robot_px_x, robot_px_y):
            # personまでの距離と座標のリストを作成
            self.target_data.clear()
            for person_px in self.person_list:
                person_point = Point()
                person_point.x = (robot_px_x - person_px.y)*self.param_dict['discrete_size']
                person_point.y = (robot_px_y - person_px.x)*self.param_dict['discrete_size']
                distance = math.sqrt(person_point.x**2 + person_point.y**2)
                self.target_data.append([distance, person_point, [person_px.x, person_px.y]])
            # 0番目に1時刻前の追従対象との距離の誤差を格納する
            self.target_data = [[self.diff_distance(data[0]), data[1], data[2]] for data in self.target_data]
            # 検出範囲内のpersonを追従対象とする(起動時だけ一番近い人を追従対象にする)
            if self.param_dict['none_person_flg']:
                target = min(self.target_data)
                param_bool = Parameter('none_person_flg', Parameter.Type.BOOL, False)
                self.set_parameters([param_bool])
            else:
                for data in self.target_data:
                    diff = self.euclidean_distance(data[1], self.before_data[1])
                    if diff <= self.param_dict['target_range']:
                        target = data
                        break
                    else:
                        target = None
            # targetがNoneだったらself.before_dataを初期化してnone_person_flgをTrueにする
            if target is None:
                #self.before_data = [0.0, 0.0, 0.0]
                #param_bool = Parameter('none_person_flg', Parameter.Type.BOOL, True)
                #self.set_parameters([param_bool])
                pass
            else:
                # 計算のために距離を保存
                self.before_data = target
            return target
    
        def generate_target(self):
            self.target_px.clear()
            # 画像の中心を算出
            robot_x = self.height / 2
            robot_y = self.width / 2
            # 追従目標を選定
            target_person = self.select_target(robot_x, robot_y)
            if not target_person is None:
                result_point = target_person[1]
                self.center_x = target_person[2][0]
                self.center_y = target_person[2][1]
            else:
                result_point = False
            return result_point
    
    
        def img_show(self, receive_msg):
            self.laser_img = self.bridge.imgmsg_to_cv2(receive_msg, desired_encoding='bgr8')
            self.height, self.width, _ = self.laser_img.shape[:3]
            # ロボットの座標をプロット
            self.plot_robot_point()
            # 追従対象の検出範囲をプロット
            if not self.param_dict['none_person_flg']:
                self.plot_target_range()
            # personがいるか判定
            if self.person_list:
                # 追従対象を生成
                target_point = self.generate_target()
                # 追従対象がいなければロボット台車を停止する
                if not target_point:
                    self.target_point.x = 0.0
                    self.target_point.y = 0.0
                    self.point_pub.publish(self.target_point)
                else:
                    robot_x = self.height / 2
                    robot_y = self.width / 2
                    # 目標座標を生成(px): 横x, 縦y
                    target_x = self.center_x
                    target_y = self.center_y + (self.param_dict['target_dist']/self.param_dict['discrete_size'])
                    self.target_px.append(target_x)
                    self.target_px.append(target_y)
                    # 目標座標を生成(m): 縦x, 横y(ロボット座標系に合わせる)
                    self.target_point.x = (robot_x - target_y)*self.param_dict['discrete_size']
                    self.target_point.y = (robot_y - target_x)*self.param_dict['discrete_size']
                    # パブリッシュ
                    self.point_pub.publish(self.target_point)
                    # グラフに描画
                    self.plot_target_point()
                    self.plot_person_point()
    
            # ros2 bag 用にトピックとして画像を配布
            img = self.bridge.cv2_to_imgmsg(self.laser_img, encoding="bgr8")
            self.img_pub.publish(img)
    
            # 画像を表示
            cv2.imshow('follow_me', self.laser_img)
            cv2.waitKey(1)
    
    
    def main():
        rclpy.init()
        node = PersonDetector()
        try:
            rclpy.spin(node)
        except KeyboardInterrupt:
            pass
        node.destroy_node()
        rclpy.shutdown()
\end{lstlisting}

ロボット台車を制御するソースコードを、ソースコード\ref{base}に示す。
\begin{lstlisting}[caption=base\_controller.py, label=base]
    import time
    import math
    import rclpy
    from rclpy.node import Node
    from rclpy.parameter import Parameter
    from rcl_interfaces.msg import SetParametersResult, ParameterEvent
    from rcl_interfaces.srv import GetParameters
    from nav_msgs.msg import Odometry
    from geometry_msgs.msg import Twist, Point
    
    
    class BaseController(Node):
        def __init__(self):
            super().__init__('base_controller')
            # Publisher
            self.pub = self.create_publisher(Twist, '/cmd_vel', 10)
            self.data_pub = self.create_publisher(Point, '/follow_me/distance_angle_data', 10)
            # Subscriber
            self.create_subscription(Point, '/follow_me/target_point', self.callback, 10)
            self.create_subscription(Odometry, '/odom', self.odom_callback, 10)
            self.create_subscription(ParameterEvent, '/parameter_events', self.param_event_callback, 10)
            # Service
            self.srv_client = self.create_client(GetParameters, '/follow_me/person_detector/get_parameters')
            while not self.srv_client.wait_for_service(timeout_sec=0.5):
                self.get_logger().info('/follow_me/laser_to_img server is not available ...')
            # Parameters
            self.declare_parameters(
                    namespace='',
                    parameters=[
                        ('tolerance', Parameter.Type.DOUBLE),
                        ('i_range', Parameter.Type.DOUBLE),
                        ('lkp', Parameter.Type.DOUBLE),
                        ('akp', Parameter.Type.DOUBLE),
                        ('aki', Parameter.Type.DOUBLE),
                        ('akd', Parameter.Type.DOUBLE)])
            self.add_on_set_parameters_callback(self.param_callback)
            # Get parameters
            self.param_dict = {}
            self.param_dict['tolerance'] = self.get_parameter('tolerance').value
            self.param_dict['i_range'] = self.get_parameter('i_range').value
            self.param_dict['lkp'] = self.get_parameter('lkp').value
            self.param_dict['akp'] = self.get_parameter('akp').value
            self.param_dict['aki'] = self.get_parameter('aki').value
            self.param_dict['akd'] = self.get_parameter('akd').value
            self.param_dict['target_radius'] = self.get_param()  # person_detectorからもってくる
            # Value
            self.twist = Twist()
            self.target_angle = 0.0
            self.target_distance = 0.0
            self.target_x = 0.0
            self.target_y = 0.0
            self.delta_t = 0.0
            self.robot_angular_vel = 0.0
            # Output
            self.output_screen()
    
        def output_screen(self):
            for key, value in self.param_dict.items():
                self.get_logger().info(f"{key}: {value}")
    
        def param_event_callback(self, receive_msg):
            for data in receive_msg.changed_parameters:
                if data.name == 'target_radius':
                    self.param_dict['target_radius'] = data.value.double_value
                    self.get_logger().info(f"Param event: {data.name} >>> {self.param_dict['target_radius']}")
    
        def param_callback(self, params):
            for param in params:
                self.param_dict[param.name] = param.value
                self.get_logger().info(f"Set param: {param.name} >>> {param.value}")
            return SetParametersResult(successful=True)
    
        def get_param(self):
            req = GetParameters.Request()
            req.names = ['target_radius']
            future = self.srv_client.call_async(req)
            while rclpy.ok():
                rclpy.spin_once(self, timeout_sec=0.1)
                if future.done():
                    break
            return future.result().values[0].double_value
    
        def point_to_angle(self, point):
            return math.degrees(math.atan2(point.y, point.x))
    
        def point_to_distance(self, point):
            distance = math.sqrt(point.x**2 + point.y**2)
            if point.x < 0.0:
                distance = -distance
            return distance
    
        def callback(self, receive_msg):
            self.target_x = receive_msg.x
            self.target_y = receive_msg.y
            self.target_distance = self.point_to_distance(receive_msg)
            self.target_angle = self.point_to_angle(receive_msg)
    
        def odom_callback(self, receive_msg):
            self.robot_angular_vel = receive_msg.twist.twist.angular.z
    
        # 比例制御量計算
        def p_control(self):
            return self.param_dict['akp']*self.target_angle
    
        # 微分制御量計算
        def d_control(self, p_term):
            return self.param_dict['akd']*(p_term - self.robot_angular_vel)
    
        # 積分制御量計算
        def i_control(self, p_term, d_term):
            value = 0.0
            diff = (p_term + d_term) - self.robot_angular_vel
    
            if not diff < self.param_dict['tolerance'] and diff < self.param_dict['i_range']:
                value = self.param_dict['aki']*self.target_angle*self.delta_t
    
            return value
    
        def pid_update(self):
            # 制御量を計算
            p_term = self.p_control()
            d_term = self.d_control(p_term)
            i_term = self.i_control(p_term, d_term)
    
            linear_vel = self.param_dict['lkp']*self.target_distance
            angular_vel = -1*(p_term + i_term + d_term)
    
            if linear_vel < 0.0:
                linear_vel = 0.0
                angular_vel = 0.0
    
            return linear_vel, angular_vel
    
        def in_range(self):
            result = False
            if abs(self.target_distance) <= self.param_dict['target_radius']:
                result = True
            return result
    
        def execute(self, rate=100):
            start_time = time.time()
            before_time = 0.0
    
            while rclpy.ok():
                self.delta_t = time.time() - start_time
                rclpy.spin_once(self)
    
                # 許容範囲内外を判
                if self.in_range():
                    linear_vel = 0.0
                    angular_vel = 0.0
                else:
                    linear_vel, angular_vel = self.pid_update()
    
                # 制御量をパブリッシュ
                self.twist.linear.x = linear_vel
                self.twist.angular.z = angular_vel
                self.pub.publish(self.twist)
    
                # 実験用に目標までの距離と角度をパブリッシュ
                data = Point()
                data.x = self.target_distance
                data.z = self.target_angle
                self.data_pub.publish(data)
    
                time.sleep(1/rate)
    
    
    def main():
        rclpy.init()
        node = BaseController()
        try:
            node.execute()
        except KeyboardInterrupt:
            pass
    
        node.destroy_node()
        rclpy.shutdown()
\end{lstlisting}

以上のソースコードをまとめて起動するソースコードを、ソースコード\ref{launch}に示す。
\begin{lstlisting}[caption=follow\_me.launch.py, label=launch]
    import os
    from ament_index_python.packages import get_package_share_directory
    import launch
    from launch import LaunchDescription
    from launch.actions import DeclareLaunchArgument
    from launch_ros.actions import Node
    
    def generate_launch_description():
        config = os.path.join(
            get_package_share_directory('recognition_by_lidar'),
            'config',
            'follow_me_params.yaml')
    
        namespace = 'follow_me'
    
        return LaunchDescription([
            Node(
                namespace=namespace,
                package='recognition_by_lidar',
                executable='laser_to_img',
                name='laser_to_img',
                parameters=[config],
                output='screen',
                respawn=True),
            Node(
                namespace=namespace,
                package='recognition_by_lidar',
                executable='person_detector',
                name='person_detector',
                parameters=[config],
                output='screen',
                respawn=True),
            Node(
                namespace=namespace,
                package='recognition_by_lidar',
                executable='base_controller',
                name='base_controller',
                parameters=[config],
                output='screen',
                respawn=True,
                on_exit=launch.actions.Shutdown()),
            ])
\end{lstlisting}